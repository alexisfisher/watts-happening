\section*{Implementation}
\subsection*{Logging}
Core to the success of Watt's Happening is the logging and structure of all application, CPU, battery, network, and hardware data that allows the follow-on analysis.  
Android OS provides SQLite for easy database creation and management. 
The initial decision on what information to collect was simply to capture as much as possible to identify correlations between applications, the hardware resources they use, and the effects on the battery.  
After understanding the trend on a few different devices, useless information can be discarded to reduce Watt's Happening's impact on the device.
To organize the data collected from the various managers provided by Android OS, a separate table was created for each major resource to include application, battery, GPS, hardware, and network status.  
Each table includes a timeslice which allows the analyzer to relate the effects of running applications to the resources used at the same time.  
This will be discussed in further detail under Section 3.
To reduce the impact of Watt's Happening on the host device, the data collection occurs as a service in the background.  
Since the service is not CPU intensive, it does not require a thread of its own and allows 
the user to run other applications without much overhead.  
To further reduce the application's battery cost, the GPS logger uses passive listening to only receive updates when another application uses the receiver.  
This piggy-backing approach paints a decent picture of device location without requiring the more expensive GPS or network provider.
An additional overhead Watt's Happening seeks to limit is the potentially unlimited growth of the database.  
The program uses a global variable called timeslice that represents the amount of time between information requests from the resource managers.  
A smaller timeslice allows for a fine granularity to identify sudden changes in CPU use by short, intense applications.  
However, a small timeslice also adds dozens of entries to each data table. 
 Over the course of several days of logging data, the database can grow to an unacceptable size.  
We reduce the memory cost of the database by condensing the data every 24 hours into a persistent table that represents the long term usage of the user.  
New data brought into the persistent table is weighted so increase the impact of recent activity into our analysis.
\subsection*{Analyzing Data}
The brunt of our project relies on the successful interpretation of the collected data.  
Ideally, one could monitor the change in battery life and determine the cost of an application.  
However, there are several complications that affect this analysis.  
Foremost, one must weigh the importance of recent data versus historic data.  
A user's device usage rate is not likely to stay constant throughout the day.  
For the majority of the day, the device might sit idle with occasional high power draining applications. 
As a result, the long term usage rate alone cannot be the sole representation for analyzing and predicting the remaining time.  
A common example would be a device that has been on idle for nearly a whole day.  
The long term rate at this point would be a slow decrease that has left the phone low on remaining battery.  
For the next five minutes, the user begins playing a CPU and network intensive game.  
The user then wants to know if their device will last another hour.  
If only the long term rate is used, that last five minute high usage rate is lost amongst the previous 24 hours.  
However, the user is likely to continue running the game in the immediate future.  
The long term collection and the most recent usage are combined with specific weights to create the most insightful recommendation possible. 
With the overall rate and predicted end time determined, the application will be able to identify the primary culprits in the battery drain.  
For our project, a culprit is an application that uses battery through intensive CPU or network use.  
When the analyzer is started, Watt's Happening cycles through the list of currently running applications.  
As stated in the assumptions section, only currently running applications are candidates for termination.  
The analyzer considers the recent behavior as captured into the database and calculates the CPU and network use designated to each UID during the duration that it has been running.  
With the CPU and network cost assigned to each culprit, the analyzer can determine the approximate cost of each culprit by combining those values with the amount of battery drained overall during the time period that the application was running.  
The resulting value is used as a heuristic variable to allow comparisons between culprits and the follow-on recommendations to the user.

\subsection*{Making Recommendations}
The key to the success of Watt's Happening is arming the user with knowledge of their use and make knowledgeable recommendation to keep the device alive.  
The logged information keep over a time period displaying the battery percentage versus time on a graph allows a graphical representation for the user to observe their usage habits. 
The rate determined in the analyzer depicts on the graph the predicted battery termination time.  
Should that time end before the user's goal, the application can display the top culprits currently running so the user is made aware of their candidates to close in order to reach their desired end time.

