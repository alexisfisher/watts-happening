\section{Related Work}
\label{sec:related}
Zwang et al.\cite{Zhang:2010:AOP:1878961.1878982} created a similar program in 2010 called PowerTutor that provides a power estimate to various hardware components and then assigns power usage to specific applications.  
They created the benchmarks using a previous project call PowerBooter that monitored power consumption while controlling power management for individual components. 
Since they base their models off of specific hardware components, their application is intrinsically tied to specific devices.
They warn that if PowerTutor is used with devices other than their target devices, ``power consumption estimates will be rough."
We intended to create a more general application that can provide useful information to a user regardless of their device's hardware. 
PowerTutor ultimately allows developers to determine the impact of their own application to improve their software designs.  
Future work could seek to incorporate aspects of PowerTutor into Watt's Happening to fine-tune application usage details and improve recommendations.

Chang, Agrawal, and Cameron survey the power management techniques employed by Android and how popular applications abuse or take advantage of those techniques\cite{energy-aware}.
They explore the idea that developers should optimize their software written for mobile devices for efficient power consumption, but in many cases the application developer does not necessarily take power into consideration. 
They perform several case studies of popular applications and analyze how they interact with the CPU, the screen, and other aspects of mobile devices. 
They came to the conclusion that even though Android does a great job aggressively saving power, application developers need to create their software with power consumption in mind. 
The paper provides a beneficial overview of the Android power management system as well as several tools that a developer can use to monitor the power usage of a device's underlying hardware. 

Pathak, Hu, and Zhang help application developers create power conserving software through their energy profiling tool called \emph{eprof}\cite{Pathak:2012:ESI:2168836.2168841}. 
The tool allows developers to monitor power consumption with multiple granularities and determine which pieces of code consume the most energy. 
This is useful for catching programming bugs, such as holding onto a wakelock too long, as well as directing attention to code that requires optimization. 
It also illuminates the fact that many applications use large amounts of energy in I/O operations. 
This implies that in order to save energy, an application should only use 3G, GPS, and WiFi when no other option exists.
These observations led us to the conclusion that \emph{Watt's Happening} needs to record and correlate network usage to specific applications.
Also, since network operations were shown to be energy intensive, a recommendation system should place an emphasis on applications that abuse the device's networking.
%Therefore, it is clear that disabling these aspects of a mobile device can make profound differences in battery life.

There has also been significant work that offers insights into user profiling and modeling.
Eventually, \emph{Watt's Happening} aspires to leverage user profiling to automatically control the power of different radios on the device. 
This would allow the application to prolong the battery life of a mobile device just by observing a user's usage patterns.
Intelligent models are necessary in order to keep the application from intruding on user actions.
If the application's actions cause annoyance to the user, they will then refuse to use the application. 

Balajinath and Raghavan discuss learning user behavior to build a model of non-malicious user actions, with the intent to easily identify malicious users\cite{Balajinath:2001:IDT:2294491.2294970}. 
They record the user's activity, and develop a newness factor to help identify behavior out of the norm. 
They identify a 500-command history as the amount of history required to ensure maximum prediction accuracy. 

Fawcett et al.\ use usage profiling to detect cellular cloning fraud \cite{dataMiningFraudDetection}. 
While the idea of profiling users based on when and where they use services is similar to what we are trying to do, they use this information for a much different application. 

Murmuria et al.\ developed a system for modeling power consumption on smartphone devices by collecting data on the wakelock drivers for a device's subsystem \cite{mobilePowerUsageMeasurements}. 
They break down the major subsystems (CPU, display, graphics, GPS, audio, and WiFi) and develop metrics by which power consumption can be determined for each subsystem. 
The paper provides a good overview of a low impact way of determining power consumption of individual components in a smartphone.
This correlation between subsystem and power consumption could allow for more powerful recommendations by \emph{Watt's Happening}.
\emph{Watt's Happening} could then give predictions stating how long a specific application can run while still achieving the overall battery-life goal for the device.

Goecks and Shavlik address unobtrusive observation of user behavior \cite{Goecks:2000:LUI:325737.325806}. 
Users do not need to necessarily identify successful results, but one can infer measurements of success from properties of successful predictions. 
These properties should be observable without user interaction to assess likelihood of success. 

Rumble et al.\ discussed a new operating system called Cinder that sought to extend limited energy resources by implementing capacitor objects \cite{Rumble:2009:AJT:1592606.1592618}. 
The similarities to their work included a desire to reach a goal usage time by dynamically running statistics of application energy consumption. 
The differences between their OS and our goal is that they built a new OS where as we aim to create an application on top of Android. 
In addition, Cinder actively limits the energy consumption of various applications where we create suggestions for the user to implement. 

Zeng et al.\ established energy as a resource called currentcy and established pricing for using that resource \cite{Zeng:2003:CUA:1247340.1247344}. 
Both currentcy and\emph{ Watt's Happening} set usage time goals and dynamically adjusts to overspent and underspent energy to allow the application to meet that energy goal. 
However, currentcy forces all devices on the platform to adopt the energy resource type. 
This makes application programmers factor currentcy in to their design whereas \emph{Watt's Happening} aims to be invisible to other applications. 

Shi et al.\ develop a method of identifying users based on behavior profiling \cite{learningUserBehavior}. 
Their methods of learning users' typical habits could be applied to power usage as well. 
If we know that a person uses their GPS when traveling to work every morning then we can account for that required power consumption when modeling usage goals. 

Oliner et al.\ developed Carat, a system to identify misbehaving code and energy-inefficient applications \cite{Oliner:2012:CED:2387858.2387864}.
The designers differentiated Carat from \emph{Watt's Happening} by using a community model.
Carat's sparse information gathering is also different than \emph{Watt's Happening}'s continuous logging cycle.
Due to this sparseness, Carat's estimations of application usage may not be available to the user for up to a week after initial installation.
\emph{Watt's Happening} can provide preliminary observations after the first analysis has been performed, as quickly as 15 minutes after initial installation.

%%%%%%%%%%%%%%%%%%%%%%%%%
%
% thoughts on carat:
% carat uses sparse information, device (assuming therefore user) specific.
% can take up to a week to provide recommendations (as opposed to WH's hours)
% good ideas to cull: 
% 	confidence rating
% 	measure of similar devices (more difficult to define 'similar' w/android)
% 	
% 
%%%%%%%%%%%%%%%%%%%%%%%%%
