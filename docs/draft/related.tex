\section*{Related Work}
\label{sec:related}
Zwang et al.\cite{Zhang:2010:AOP:1878961.1878982} created a similar program in 2010 called PowerTutor that receives a power estimate to various hardware components and application and then assigns power usage to specific applications.  
They created the benchmarks using a previous project call PowerBooter that monitored power consumption while controlling power management for individual components.  
PowerTutor ultimately allows developers to determine the impact of their own application to improve their software designs.  
Future work could seek to incorporate PowerTutor into Watt's Happening to fine-tune application usage details and improve recommendations.

Chang, Agrawal, and Cameron survey the power management techniques employed by Android and how popular applications abuse or take advantage of those techniques\cite{energy-aware}. 
They explore the idea that software written for mobile devices should be optimized for efficient power consumption, but in many cases the application developer doesn't necessarily take power into consideration. 
They do several case studies of popular applications and analyze how they interact with the CPU, the screen, and other aspects of the mobile devices. 
They came to the conclusion that even though Android does a great job to aggressively save power, application developers need to create their software with power consumption in mind. 
The paper is a beneficial overview of the Android power management system as well as several tools that can be used to monitor the power usage of a device's underlying hardware. 

Pathak, Hu, and Zhang help application developers create power conserving software through their energy profiling tool called \emph{eprof}\cite{Pathak:2012:ESI:2168836.2168841}. 
The tool allows developers to monitor power consumption with multiple granularities and determine which pieces of code consume the most energy. 
This is useful for catching programming bugs, such as holding onto a wakelock too long, as well as directing attention to code that needs to be optimized. 
It also illuminates the fact that large amounts of energy are used in I/O operations. 
This implies that in order to save energy it is important to only use 3G, GPS, and WiFi when it is truly necessary. 
Therefore, it is clear that disabling these aspects of a mobile device can make profound differences in battery life.

Balajinath and Raghavan discuss learning user behavior to build a model of non-malicious user actions, with the intent to easily identify malicious users\cite{Balajinath:2001:IDT:2294491.2294970}. 
User activity is recorded, and a newness factor is used to help identify behavior out of the norm. 
A 500-command history was identified as the amount of history required to ensure maximum prediction accuracy. 

Fawcett et al. use usage profiling to detect cellular cloning fraud \cite{dataMiningFraudDetection}. 
While the idea of profiling users based on when and where they use services is similar to what we are trying to do, they are using this information for a much different application. 

Ashbrook learns significant locations and predicts user movement from GPS data \cite{Ashbrook:2002:LSL:862896.881068}. 
Raw location data is analyzed using k-clustering with a variety of radiuses. 
A building location is inferred from lack of GPS information for an arbitrarily selected ten minute time slice. 
This is vulnerable to other GPS data issues, including urban canyons or device failure. 
The location prediction is based on a Markov model built off of observed patterns. 
First order Markov models are used, as data models beyond that would require a much longer observation period. 

Goecks and Shavlik address unobtrusive observation of user behavior \cite{Goecks:2000:LUI:325737.325806}. 
Users don't need to necessarily identify successful results, but measurements of success can be inferred from properties of successful predictions. 
These properties should be observable without user interaction to assess likelihood of success. 

Murmuria et al. develop a system for modeling power consumption on smartphone devices by collecting data on the wakelock drivers for a device's subsystem \cite{mobilePowerUsageMeasurements}. 
They break down the major subsystems (CPU, display, graphics, GPS, audio, and WiFi) and develop metrics by which power consumption can be determined for each subsystem. 
This paper provides a good overview of a low impact way of determining power consumption of individual components in a smartphone. 

Rumble et al. discussed a new operating system called Cinder that sought to extend limited energy resources by implementing capacitor objects \cite{Rumble:2009:AJT:1592606.1592618}. 
The similarities to their work included a desire to reach a goal usage time by dynamically running statistics of application energy consumption. 
The differences between their OS and our goal is that they built a new OS where as we aim to create an application on top of Android. 
In addition, Cinder actively limits the energy consumption of various applications where we will create suggestions for the user to implement. 

Zeng et al. established energy as a resource called currentcy and established pricing for using that resource \cite{Zeng:2003:CUA:1247340.1247344}. 
Both currentcy and Watt's Happening set usage time goals and dynamically adjusts to overspent and underspent energy to allow the application to meet that energy goal. 
However, currentcy forces all devices on the platform to adopt the energy resource type. 
This makes application programmers factor currentcy in to their design whereas Watt's Happening aims to be invisible to other applications. 

Shi et al. develop a method of identifying users based on behavior profiling \cite{learningUserBehavior}. 
Their methods of learning users' typical habits could be applied to power usage as well. 
If we know that a person uses their GPS when traveling to work every morning then we can account for that required power consumption when modeling usage goals. 


%%%%%%%%%%%%%%%%%%%%%%%%%
%
% thoughts on carat:
% carat uses sparse information, device (assuming therefore user) specific.
% iOS is a much narrower range of devices & OSes
% can take up to a week to provide recommendations (as opposed to WH's hours)
% good ideas to cull: 
% 	confidence rating
% 	measure of similar devices (more difficult to define 'similar' w/android)
% 	
% 
%%%%%%%%%%%%%%%%%%%%%%%%%