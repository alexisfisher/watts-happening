Rumble Et Al discussed a new operating system called Cinder that sought to extend limited energy resources by implementing capacitor objects.  The similarities to
their work included a desire to reach a goal usage time by dynamically running statistics of application energy consumption.  The differences between their OS and our
goal is that they built a new OS where we aim to create an application on top of Android.  In addition, Cinder actively limits the energy consumption of various
applications where we will create suggestions for the user to implement.

Zeng Et Al established energy as a resource called currentcy and established pricing for using that resource.  Both currentcy and Watt's Happening set usage
time goals and dynamically adjusts to overspent and underspent energy to allow the application to meet that energy goal.  However, currentcy forces all
devices on the platform to adopt the energy resource type.  This makes application programmers factor currentcy in to their design whereas Watt's Happening
aims to be invisible to other applications.
