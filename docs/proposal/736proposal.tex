\documentclass{article}
\usepackage{alltt}
\usepackage{verbatim}

\title{Project Proposal: Watt's Happening}
\author{Ben Bramble, Nick Burek, Alexis Fisher, and Adam Vail}
\begin{document}
\maketitle

\section*{Problem}
Mobile devices, including smart phones and tablets, are capable of performing an ever-increasing array of tasks. 
Unfortunately, short battery life severely limits the full potential of smart phones. 
A user may experience erratic power surges based on different applications and might not know if her phone will last until the end of the day. 
While power analyzing applications exist, most are intended for use by energy-conscious application developers but not for end users \cite{eprof} \cite{energy-aware}. 
Of the power saving applications available for end users, some complain of the applications being too intrusive. 
These applications disable important features of the phone without user consent and make practical usage very difficult. 
Allowing the user to set a target ``no-charge-until" goal time, and then guiding the user towards decisions to achieve this goal, would enable more efficient and productive use of mobile devices.

\section*{Motivation}
Mobile devices are nearly ubiquitous, and function as an integral part of modern life. 
As battery technology lags behind the ever-increasing demands applications make, devices need intelligent use of limited resources. 
Unfortunately, most application developers do not take into account these power limitations, or even realize the power draws of their applications on disparate devices. 
Compounding this problem, common users remain blissfully unaware of their power use until their phone reaches a minimum battery threshold.

\section*{Proposed Project}
Intelligently learning a user's standard behavior and performing power-saving adjustments that fit with the user's needs would enable longer battery life. 
Adjustments could include powering down radios (including gps, wifi, 4g, 3g) automatically, terminating applications that are drawing extreme power without user interaction, 
and recommending user application termination when the goal may not be achievable based on current usage. 
The proposed end state is an easy-to-use application that educates the user based on their individual energy use and allows the user to customize their mobile device to extend the battery life on a single charge.

\section*{Specific Tasks}
\begin{itemize}
\item Gather battery \& usage information in order to analyze user's usage habits.
	\begin{itemize}
	\item Develop initial heuristics for different application radio footprints to advise on previously unseen applications.
	\end{itemize}
\item Receive user's usage goal in order to allow ``Watt's Happening'' application to advise user.
\item Present user with list of recommendations and effects to allow user to manage device power.
\item Continue monitoring battery \& usage information to dynamically make new recommendations.
	\begin{itemize}
	\item Watt's Happening application has whitelist/blacklist of applications and power devices to provide automatic power management.
	\end{itemize}
\end{itemize}

\subsection*{Methodology}
Our goal is to optimize the overall battery usage of the entire device. 
Therefore, we plan to monitor a variety of usage scenarios, gathering battery and usage snapshots throughout the test period. 
An approximate duplication of these usage scenarios using our ``Watt's Happening" application should provide information on general power trends. 
Ideally, the use of ``Watt's Happening" will provide a noticeable extension of battery life without impacting necessary or desired user operations.

Although formal and repeatable testing would provide a consistent and controlled environment to display incontrovertible usage trends, a sterile environment is not the expected use case.
Therefore, averaging real world usage scenarios will hopefully provide a more realistic, yet still convincing, data set.

\subsection*{Resources}
\begin{itemize}
\item Android SDK
\item Android device. (Nick's Android phone, Professor Banerjee has several tablets that run Android as well)
\end{itemize}

\begin{thebibliography}{9}

\bibitem{eprof}
Abhinav Pathak, Y. Charlie Hu, and Ming Zhang,
  ``Fine Grained Energy Accounting on Smartphones with Eprof,"
  in \emph{Proc. of EuroSys}, 2012. \\ 
  URL: \verb https://engineering.purdue.edu/~ychu/publications/eurosys12_eprof.pdf .

\bibitem{energy-aware}
Hung-Ching Chang, Abhishek R Agrawal, and Kirk W Cameron,
  ``Energy-Aware Computing for Android Platforms," \\
  URL: \verb http://scape.cs.vt.edu/wp-content/uploads/2012/08/ITJ12_Android_Energy-Aware.pdf .

\end{thebibliography}

\end{document}
